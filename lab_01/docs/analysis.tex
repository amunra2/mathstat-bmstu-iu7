\chapter{Задание на лабораторную работу}

\textbf{Цель работы:} построение гистограммы и эмпирической функции распределения.

\section{Содержание работы}

\begin{enumerate}
    \item Для выборки объема \textit{n} из генеральной совокупности \textit{X} реализовать в виде программы на ЭВМ

    \begin{enumerate}
        \item вычисление максимального значения $M_{max}$ и минимального значения $M_{min}$;
        \item размаха $R$ выборки;
        \item вычисление оценок $\hat\mu$ и $S^2$ математического ожидания $MX$ и дисперсии $DX$;
        \item группировку значений выборки в $m = [\log_2 n] + 2$ интервала;
        \item построение на одной координатной плоскости гистограммы и графика функции плотности распределения вероятностей нормальной случайной величины с математическим ожиданием $\hat{\mu}$ и дисперсией $S^2$
        \item построение на другой координатной плоскости графика эмпирической функции распределения и функции распределения нормальной случайной величины с математическим ожиданием $\hat{\mu}$ и дисперсией $S^2$.
    \end{enumerate}

    \item Провести вычисления и построить графики для выборки из индивидуального варианта.
\end{enumerate}


\chapter{Теоретическая часть}

\section{Формулы для вычисления величин}

Максимальное $M_{max}$ и минимальное $M_{min}$ значения выборки:

\begin{equation}
    \begin{aligned}
        M_{\max} = X_{(n)}\\
        M_{\min} = X_{(1)}
    \end{aligned}
\end{equation}


Размах $R$ выборки:

\begin{equation}
    R = M_{\max} - M_{\min}.
\end{equation}


Оценки математического ожидания и дисперсии:

\begin{itemize}
    \item выборочное среднее:

    \begin{equation}
        \begin{aligned}
        \hat\mu(\vec X_n) &= \frac 1n \sum_{i=1}^n X_i\\
        \end{aligned}
    \end{equation}
    
    \item несмещенная оценка дисперсии:

    \begin{equation}
        \begin{aligned}
        S^2(\vec X_n) &= \frac 1{n-1} \sum_{i=1}^n (X_i-\overline X_n)^2
        \end{aligned}
    \end{equation}

\end{itemize}


\section{Определение эмпирической плотности и гистограммы}

\subsubsection{Интервальный статистический ряд}

\textbf{\underline{Определение}}: Интервальным статистическим рядом, отвечающим выборке $\vec x$, называется таблица вида:

\begin{table}[htb]
    \centering
    \begin{tabular}{|c|c|c|c|c|}
        \hline
        $J_1$ & $J_2$ & ... & $J_m$ \\
        \hline
        $n_1$ & $n_2$ & ... & $n_m$ \\
        \hline
    \end{tabular}
\end{table}

Здесь $n_i$ --- число элементоав выборки $\vec x$, попавших в промежуток $J_i$, i = $\overline{1, m}$, где

\begin{equation}
    J_i = [x_{(1)} + (i - 1) \cdot \Delta, x_{(1)} + i \cdot \Delta), i = \overline{1; m - 1}
\end{equation}

Причем:

\begin{equation}
    J_{m} = [x_{(1)} + (m - 1) \cdot \Delta, x_{(n)}]
\end{equation}

Величина $\Delta$ при этом равна:

\begin{equation}
    \Delta = \frac{|J|}{m} = \frac{x_{(n)} - x_{(1)}}{m}
\end{equation}


\subsubsection{Эмперическая плотность}

Пустя для данной выборки $\vec x$ построен интервальный статистический ряд $(J_i, n_i)$, $i = \overline{1; m}$\newline

\textbf{\underline{Определение}}: Эмпирической плотностью распределния (соответствующей выборке $\vec x$) называется функция:

\begin{equation}
    f_n(x) =
    \begin{cases}
        \frac{n_i}{n \cdot \Delta}, x \in J_i, i = \overline{1; m} \\
        0, \text{иначе} \\
    \end{cases}
\end{equation}


\subsubsection{Гистограмма}

\textbf{\underline{Определение}}: График эмпирической функции плотности называется гистограммой.


\section{Определение эмпирической функции распределения}

Пусть $\vec x = (x_1, ..., x_n)$ --- выборка из генеральной совокупности $X$.

Обозначим $n(t, \vec x)$ --- числа компонент вектора $\vec x$, которые меньше, чем $t$.\newline

\textbf{\underline{Определение}}: Эмпирической функцией распределения, построенной по выборке $\vec x$, называют функцию

\begin{equation}
    F_n: {R} \to {R},
\end{equation}

определенную правилом:

\begin{equation}
    F_n(x) = \frac{n(t, \vec x)}{n}
\end{equation}



\chapter{Практическая часть}

\section{Код программы}

\lstinputlisting{../src/main.m}


\section{Результат работы программы}

\subsubsection{Вывод программы}

\begin{lstlisting}
Выполнение заданий.


1) Вычислить максимальное и минимальное значение:

Минимальное значение  = -17.2900
Максимальное значение = -12.9100

2) Вычислить размах R:

R = 4.3800

3) Вычислить оценки математического ожидания и дисперсии:

Выборочное среднее (оценка мат ожидания) = -15.2209
Исправленная дисперсия (оценка дисперсии) = 0.9680

4) Группировка значений выборки в m = [log2 n] + 2 интервала:

Интервалов m =   8

    1. [ -17.2900 ; -16.7425) : Элементов:   9
    2. [ -16.7425 ; -16.1950) : Элементов:  13
    3. [ -16.1950 ; -15.6475) : Элементов:  19
    4. [ -15.6475 ; -15.1000) : Элементов:  22
    5. [ -15.1000 ; -14.5525) : Элементов:  21
    6. [ -14.5525 ; -14.0050) : Элементов:  23
    7. [ -14.0050 ; -13.4575) : Элементов:  11
    8. [ -13.4575 ; -12.9100] : Элементов:   2

5) Построить гистограмму и график функции плотности
    распределния вероятностей нормальной случайной величины с
    математическим ожиданием mu и s_quad

Результат в отдельном окне

6) Построить график эмпирической функции распределения и
    функции распределения нормальной случайной величины с
    математическим ожиданием mu и s_quad

Результат в отдельном окне
\end{lstlisting}

\imgHeight{90mm}{gisto}{Гистограмма и график функции плотности распределения нормальной случайной величины с выборочными математическим ожиданием и дисперсией}

\imgHeight{90mm}{emperic}{График эмперической функции распределения и функции распределения нормальной случайной величины с выборочными математическим ожиданием и дисперсией}
